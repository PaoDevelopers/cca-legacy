\input{header.texinc}

\title{Criterion A: Planning}

\begin{document}
\maketitle

\section{The scenario}


My client, \textbf{Mr.~F} is the head of the Co-Curricular Activities
(CCA) department. He approached me knowing that I was good at programming and
wanted to know if I could help him sort out the problems he and the department
was having with SchoolBuddy, the school's current CCA admin tool.  At our first
meeting Mr F explained tyhe issues with the current system and how it
creates a lot of additional paperwork for the department how it usually crashes
at the start of each new session due to limitation on the technical side,
please see addendix 1, meeting 1 for full details.  After this meeting I looked
into the problem and did some reseach on how 

Second meeting: ideas
Third meeting: Finalization of functions, agree on success criterias

SchoolsBuddy
presents the following challenges for him, students, and the school financial
department:

\begin{itemize}
	\item SchoolsBuddy allows students select CCAs that are already full.
		CCA staff must then contact these students manually and tell
		them to choose again (usually from a more limited course pool
		since all the popular ones would have been full at this point).
	\item The selection system does not enforce CCA hours requirements, and
		the staff must manually verify that students have completed CCA
		choices to the year group's requirements by printing out the
		spreadsheet to paper and reading through them.
	\item It utilizes PowerSchool single sign-on, which only allows
		approximately 300 sessions at once.
	\item The web page is exremely bloated at approximately 16.5\,MiB
		gzipped each. It takes more than a minute to load during the
		congestion that occurs when everyone is choosing CCAs.
	\item Some parts of the interface are unintuitive to students.
	\item The school has to pay SchoolsBuddy an expensive subscription fee.
\end{itemize}

\section{Rationale for the proposed solution}

The IT department confirmed that they could deploy a custom solution on our
school LAN when given the go-ahead from Mr.~Funnel.

I am relatively experienced in developing low-latency network applications, I
am capable of optimizing concurrent networking programs, and I am familiar with
our school's Microsoft Entra ID authentication system.

% The program does not need input data during the development process. During
% production, all data is automatically retreived from Microsoft Entra ID and the
% Microsoft Graph API via delegated access once a student has logged in via OAuth
% 2.0; in practice, this data includes the year group (grade level), name,
% student number, and email address, all of which are publicly available to any
% student via the Microsoft Entra ID portal.

There are no special security considerations other than various standard ones
present when working with web applications. Care must be taken not to leak
client secrets used in the OAuth 2.0 authorization code flow. Cookies must be
protected against cross-origin request forgery and should have
\texttt{httponly} and \texttt{secure} flags. It should be made impossible for a
student to spoof another student's course choices, provided that the victim's
school login credentials haven't been already leaked.

\section{Success criteria}

The product shall present to students and administrators an accessible and
easy-to-use web interface for choosing and managing CCAs. It shall address each
of the issues of SchoolsBuddy as presented above.

\begin{itemize}
	\item When too many students attempt to choose one course at a time,
		their attempts are sequentially processed, and those that
		exceed the CCA's member limit are properly rejected. \item It
		should be possible to log in via Microsoft Entra ID, or
		otherwise integrate to the school's login system.
	\item The web page must be lightweight. Pages should not be over 100
		KiB, uncompressed, but after minification.
	\item The course selection categories shall be relatively intuitive.
	\item The selection system must enforce CCA hours requirements.
	\item The selection system must be able to take a CSV of CCAs with
		fields such as period, location, teacher, and member cap. It
		must then be able to populate its own CCAs table that it
		presents to the students with information in the CSV.
	\item The selection system must be able to export student choices as a
		CSV containing the following fields:
		\begin{itemize}
			\item Student name
			\item Numeric student ID
			\item Student year group (grade level)
			\item CCA name
			\item CCA period
		\end{itemize}
		These may then be used to trivially import choices to
		PowerSchool. It would be best if the PowerSchool API could be
		directly accessed to insert the courses, but the API is not
		publicly documented.
\end{itemize}

\end{document}
