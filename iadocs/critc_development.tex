\input{header.texinc}

\title{Criterion C: Development}

\begin{document}
\maketitle

% Remember to demonstrate algorithmic thinking.

\section{State propagation}

On the server side, the authoratative source of state is the
PostgreSQL database. Since there should only be one instance of the
CCA selection system running on a particular database at a time, we
may carefully duplicate part of the database state into memory for
faster access and atomic operations. Memory-mapped databases such
as LMDB have been considered but were ruled out to complex
concurrency issues and poor B-tree write performance for multiple
small writes.

Note that in the following I will be referring to ``goroutines'';
those are basically lightweight threads that the Go runtime
dynamically schedules onto a number of operating system threads while
being aware of blocking channel operations and other synchronization
primitives. Each thread has its own stack which starts at about 2
kilobytes. Other than basic synchronization primitives such as
atomics, mutexes, and read-write mutexes, Go also has a useful data
type called a ``channel''; channels convey a message type of a fixed
length and has an unsigned integer representing the ``buffer length'';
it either blocks until there is a corresponding writer or reader or
fails when reading from an empty channel or writing to a full channel,
depending on how it is used.

The primary reason for needing state propagation is to update each
user's browser with information on how many seats in a CCA have been
already taken. For each CCA, this ``member counter'' is saved in
memory as an unsigned integer, and is represented in the database as
the number of rows in the ``choices'' table where the course ID
matches that of a particular CCA.

To update and propagate this state, my first attempt was to create
a read-write mutex corresponding to each course's member counter.
The main loop that handles the WebSocket connection of each user
creates a 

When a user selects or deselects a course, its member counter is
checked against the maximum number of people allowed for that CCA,
and the member counter is incremented by 1 if there are still seats
available. 

Every time someone chooses a course,
the server locks the mutex corresponding to that CCA's member counter,
which guarantees that there are no other writers. The number is read
normally (i.e., without atomics), and is then atomically incremented
by 1 


\section{Acknowledgements of existing tools}

The following tools, libraries, and other materials were used in the
development of this product.

\begin{itemize}
\item The \href{https://go.dev}{Go programming language}'s
\href{https://go.dev/ref/spec}{specification} was referred to,
especially for documentation on how channel operations work; its GC
toolchain (the most widely-used reference implementation) is used as
the compiler during development and production; its standard library
is used extensively in the program.

\item \href{https://gobyexample.com}{Go by Example} was referred to for
documentation on command-line flags and contexts.

\item \href{https://developer.mozilla.org/en-US/}{MDN Web Docs} was
used as my primary source of JavaScript documentation.

\item \href{https://godocs.io}{A hosted fork of gddo} was used as my
primary source of Go documentation, along with using the
\texttt{go doc} command as part of the GC toolchain.

\item \href{https://git.sr.ht/~emersion/go-scfg}{scfg}
written by \href{https://emersion.fr}{Simon Ser} is used to parse
configuration files.

\item \href{https://github.com/MicahParks/keyfunc}{keyfunc}
written by \href{https://micahparks.com/}{Micah Parks} is used to
update the JSON Web Key Set to validate JSON Web Tokens for user
authentication.

\item \href{https://github.com/coder/websocket}{websocket} maintained
by \href{https://coder.com/}{coder} is used for bi-directional
communication.

\item A minimal variant of the
\href{https://www.rfc-editor.org/rfc/rfc1459#section-2.3}{RFC1459
IRC message format} is used as the message format in client-server
communication.

\item \href{https://github.com/golang-jwt/jwt}{golang-jwt} is used to
parse and validate JSON Web Tokens for user authentication.

\item \href{https://github.com/google/uuid}{uuid} by Google is used to
generate
\href{https://www.rfc-editor.org/rfc/rfc9562.html}{UUIDs}
during testing.

\item \href{https://github.com/jackc/pgx}{pgx} by
\href{https://jackchristensen.com/}{Jack Christensen} is used
to establish a connection with the PostgreSQL database backend.

\item \href{https://www.postgresql.org/}{PostgreSQL} is used as a
database backend.

\item \href{https://golangci-lint.run/}{golangci-lint} is used as a
linter to detect programming errors.

\item \href{https://neovim.io/}{neovim} is used as a text editor; its
LSP client was used to connect to
\href{https://pkg.go.dev/golang.org/x/tools/gopls}{gopls} for
error detection and documentation provision while editing code.

\end{itemize}


\end{document}
